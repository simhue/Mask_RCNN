\chapter{Fazit}\label{chap:summary}

Ziel der Arbeit war es ein heuristisches Modell zu trainieren, das in der Lage ist anhand von Satellitenaufnahmen infizierte Agrarfläche zu analysieren und eventuelle Krankheitsbefälle zu erkennen. 
\\\\
Das Skript, das im Rahmen dieser Abschlussarbeit entwickelt wurde, fasst mehrere Disziplinen zusammen. Deswegen war eine ausführliche Ausarbeitung in Thematiken wie pflanzliche Biologie, Lichtspektren, geografische Referenzsysteme und künstliche neuronale Netze notwendig, um ein tiefgehendes Verständnis für die Problematik und für den Zusammenhang zwischen Lichtreflexion einer Pflanze und deren Stress- bzw. Gesundheitsstatus zu erlangen. Gesunde bzw. gestresste Pflanzen besitzen einen hohen bzw. niedrigen Chlorophyllspiegel und dieser sorgt dafür, dass die Pflanzen im nahen Infrarotbereich stärker bzw. schwächer zurückstrahlen. Diese Reflexionen können über Sentinel-2-Multispektralaufnahmen gemessen werden und daraus lässt sich ein Vitalitätsindikator NDVI berechnen.
\\\\
Die berechneten NDVI-Werte sollten in einem bewachten Lernverfahren ein neuronales Netz trainieren. Jedoch war früh ersichtlich, dass historische Daten rar sind und dadurch war auch mit Overfitting zu rechnen. Es wurden mehrere Methoden untersucht, um einem Overfitting entgegen zu wirken wie etwa Vergrößerung des Datensatzes durch Data Augmentation oder Hinzufügen eines weiteren Hyperparameters durch Ridge Regression. Um die 
\\\\
Als zu trainierendes KNN wurde die Mask R-CNN-Implementierung ausgewählt, da es unter anderem eine hohe Robustheit aufweist und Objektinstanzen auf Pixelebene klassifizieren kann. 
\\\\
Nachdem Modell und Methoden zur Modellevaluation und gegen Overfitting ausgesucht wurden, wurde ein Prozessablauf ausgearbeitet, der automatisch Sentinelprodukte in einem vorherdefinierten Zeitraum und einer geographischen Umgebung herunterlädt, verarbeitet und das Training bzw. Detektionen ausführt. Darauf wurde der konzeptierte Prozess implementiert und das Modell durch anschließende Experimente evaluiert.
\\\\
Die Experimente zeigten positive Ergebnisse durch Data Augmentation. Die anfangs trainierten Modelle ohne Data Augmentation lieferten bei leichten Veränderungen der bekannten Daten wie Rotationen oder Spiegelungen keine brauchbaren Resultate. Währenddessen sind mit Data Augmentation trainierte Modelle dagegen resistenter. Allerdings zeigte die L2 Regularization keinerlei Auswirkung.
\\\\
Es ist möglich, mit der Mask R-CNN-Architektur ein Modell trainieren zu können, das zuverlässig die hier untersuchten Krankheiten erkennt und eingrenzen kann. Fremde Pathogene werden von diesem Modell wahrscheinlich nicht erkannt, da jede Krankheit und folglich die Auswirkung auf die Pflanze unterschiedlich ist. Es sind weitere Untersuchungen mit zusätzlichen Daten notwendig, die zum Abschluss der Arbeit nicht verfügbar waren. Zum einen um die in Kapitel \ref{sec:experiment-3} beschriebenen Schwankungen ausgleichen zu können. Zum anderen um einen Datensatz zu haben, der ein fremdes mit der selben Krankheit infiziertes Feld enthält, der dazu dient die hier erlangten Ergebnisse zu bestätigen. 

\chapter{Ausblick}

Gegen Ende der Ausarbeitung gaben Mitarbeiter des CREA bekannt, im Laufe des Jahres 2019 weitere Felder auf Infektionen untersuchen zu wollen und diese entsprechend weiterzuleiten. Auf Basis dieser Daten können zukünftig weitere Experimente definiert, durchgeführt und evaluiert werden. Um mehr Satellitendaten in einem Zeitraum zu erhalten, können andere Satelliten wie etwa SPOT-5 angesprochen werden. Wichtig hierbei ist, dass die Satelliten eine gleiche oder höhere räumliche Auflösung besitzen und Aufnahmen im roten und nahen infraroten Bereich machen können. 
