\documentclass[a4paper,12pt,headsepline]{report}

%----------------- PDF CONFIG ----------------- %
\pdfinfo{    
     /Title (PDF-Titel) 
     /Subject   (PDF-Thema)    
     /Author  (Vorname Nachname) 
     /Keywords   (Stichwort1,Stichwort2)      
} 

\title{MyTitle}
\author{theAuthor}
\date{1.1.2000}



%----------------- PAKETE INKLUDIEREN ----------------- %

\usepackage{geometry} % Packet für Seitenrandabständex und Einstellung für Seitenränder
\usepackage[ngerman]{babel} % deutsche Silbentrennung

\usepackage{todonotes}
\usepackage{booktabs} %entzerrt die Tabellenzeilen und bietet verschieden dicke Unterteilungslinien
\usepackage{longtable} % Tabellen können sich nicht über mehrere Seiten 
\usepackage{graphicx} % kann LaTeX Grafiken einbinden
\usepackage{abstract}

\usepackage{amsmath}
\usepackage{gensymb}
\usepackage{siunitx}
\usepackage[utf8]{inputenc} % Umlaute unter Mac werden automatisch gesetzt
\usepackage[T1]{fontenc} % Zeichenencoding
\usepackage{lmodern} % typographische Qualität 
\frenchspacing % Schaltet den zusätzlichen Zwischenraum ab
\usepackage{fix-cm}
\usepackage{hyperref} % verwandelt alle Kapitelüberschriften, Verweise aufs Literaturverzeichnis und andere Querverweise in PDF-Hyperlinks
\usepackage{color}
\PassOptionsToPackage{hyphens}{url}

% Für Literaturnachweise
\usepackage{csquotes}
%\usepackage[style=amsalpha,backend=biber]{biblatex}
%\addbibresource{literature.bib}

\usepackage{bera}% optional: just to have a nice mono-spaced font
\usepackage{listings}
\usepackage{xcolor}

\colorlet{punct}{red!60!black}
\definecolor{background}{HTML}{EEEEEE}
\definecolor{delim}{RGB}{20,105,176}
\colorlet{numb}{magenta!60!black}

\lstdefinelanguage{json}{
    basicstyle=\normalfont\ttfamily,
    numbers=left,
    numberstyle=\scriptsize,
    stepnumber=1,
    numbersep=8pt,
    showstringspaces=false,
    breaklines=true,
    frame=lines,
    backgroundcolor=\color{background},
    literate=
     *{0}{{{\color{numb}0}}}{1}
      {1}{{{\color{numb}1}}}{1}
      {2}{{{\color{numb}2}}}{1}
      {3}{{{\color{numb}3}}}{1}
      {4}{{{\color{numb}4}}}{1}
      {5}{{{\color{numb}5}}}{1}
      {6}{{{\color{numb}6}}}{1}
      {7}{{{\color{numb}7}}}{1}
      {8}{{{\color{numb}8}}}{1}
      {9}{{{\color{numb}9}}}{1}
      {:}{{{\color{punct}{:}}}}{1}
      {,}{{{\color{punct}{,}}}}{1}
      {\{}{{{\color{delim}{\{}}}}{1}
      {\}}{{{\color{delim}{\}}}}}{1}
      {[}{{{\color{delim}{[}}}}{1}
      {]}{{{\color{delim}{]}}}}{1},
}


\usepackage[nottoc]{tocbibind}

% für Listings
\usepackage{listings}
\lstset{numbers=left, numberstyle=\tiny, numbersep=5pt, stepnumber=4, keywordstyle=\color{black}\bfseries\itshape, stringstyle=\ttfamily,showstringspaces=false,basicstyle=\footnotesize,captionpos=b}
\lstset{language=python}


%----------------- FARBEN DEFINIEREN ----------------- %
\definecolor{gray}{gray}{0.95} % Listingsbackground

%----------------- LAYOUT SETZEN ----------------- %
\geometry{left=3cm, right=3cm, top=3cm, bottom=3cm}
\linespread {1.25}\selectfont %1.25 da er von Haus aus 1.2 ist und 1,25 * 1,2 = 1,5 isch




%-------##-------##-------##------- ANFANG INHALT -------##-------##-------##-------%
\begin{document}


%----------------- DECKBLATT -----------------%
%----------------- KONFIGURATION ----------------- %
\pagestyle{empty} % enthalten keinerlei Kopf oder Fuß 

%----------------- Uni Leipzig Logo ----------------- %
\begin{figure}[t]
	\centering
	\includegraphics[width=0.6\textwidth]{pics/logo_uni_leipzig}
\end{figure}

%----------------- INHALT ----------------- %

\begin{center}
\Large Universität Leipzig \\
\normalsize Fakultät für Mathematik und Informatik\\

% Whitespace
\vspace{105 pt}

\Huge Erkennung von erkrankten Nutzpflanzen anhand von Sentinel-2-Multispektralaufnahmen \\ 
\normalsize
\vspace{20 pt}

Abschlussarbeit zur Erlangung des akademischen Grades \\ 
Master of Science (M.Sc.) 

\vspace{60 pt}


vorgelegt von \\
\vspace{5 pt}
Simon Hüning 
\vspace{100 pt}

\begin{tabular}[h]{p{4cm}l l}
	Referent: & Prof. Dr. Martin Middendorf
\end{tabular}


\end{center}
 
 %----------------- ABSTRAKT -----------------%
\begin{abstract}
\noindent
Pathogene verursachen weltweit enorme Einbußen von Ernteerträgen. Zum Beispiel kann aus einer Anthraknose-Infektion ein Ertragsverlust von bis zu 50 \% resultieren. Methoden zur Bekämpfungen und Prävention wie Pestizide von Erkrankungen belasten nicht nur die Landwirte monetär, sondern auch durch die Verteilung von Chemikalien die Umwelt schädigen. Daraus folgende manuelle Kontrollen sind zeitaufwendig. Die Sentinel-2-Satelliten erzeugen regelmäßige Multispektralaufnahmen, aus denen ein Normalized Difference Vegetation Index (NDVI) berechnet werden kann. Der NDVI ist ein Indikator für den Vitalitätsstatus der Nutzpflanzen. In dieser Arbeit wurde das Mask Region-based Convolutional Neural Network (Mask R-CNN) untersucht, ob es eine Nutzpflanzen-Infektion anhand der NDVI-Werte identifizieren kann. Da nur wenige Daten über Infektionen vorhanden sind, wurden ebenfalls Data Augmentation und L2 Regularization zur Vermeidung von Overfitting evaluiert. Es hat sich gezeigt, dass Data Augmentation eine nützliche Methode gegen Overfitting ist. Darüber hinaus hilft es durch Randomisierungen des Datensatzes zur Generalisierung des Mask R-CNN. Obwohl Mask R-CNN und Data Augmentation potentielle Werkzeuge sind, um Agrarflächen zu analysieren, sind Forschungen mit weiteren Daten von erkrankten Feldern notwendig.
\end{abstract}
 
%----------------- VERZEICHNISSE -----------------%
\tableofcontents % Inhaltverzeichnis

%----------------- INHALT -----------------%
\chapter{Methoden}\label{chap:methods}

Bevor ein künstliches neuronales Netzwerk trainiert werden kann, müssen Daten vorbereitet werden. In diesem Kapitel wird der Prozess von der Generierung der Trainingsdaten über deren Vorverarbeitung bis hin zum Trainingsschritt beschrieben.

\section{Trainingsdaten}\label{sec:data}
\begin{figure}[ht]
  \centering
  \includegraphics[width=.7\textwidth]{pics/roi.png}
  \caption[Region of Interest]{RGB-Sentinel-2-Aufnahme des zu untersuchenden Ackers. Die infizierten Flächen sind weiß umrandet. }
  \label{fig:roi}
\end{figure}
\noindent
Der infizierte Acker, der die Basis des Datensatzes bildet, befindet sich etwa $15$ km nordwestlich von Bologna in Norditalien (\ang{44;34;28,92} Nord, \ang{11;10;21,36} Ost). Das Feld hat eine Fläche von $7640,57$ m$^2$ und ist mit Sorghum bepflanzt. Mitarbeiter des CREA (Council for Agricultural Research and Economics) haben vor Ort am 12.07.2018 Befälle von Anthracnose und der bakteriellen Streifenkrankheit (med.: Xanthomonas translucens) diagnostiziert. Etwa die Hälfte der Pflanzen des Feldes sind betroffen. Dabei sind im östlichen Teil des Feldes (Abb. \ref{fig:roi}, innere Markierung) auf einer Fläche von $1043,22$ m$^2$ von etwa $60$ bis $70\%$ der Pflanzen befallen.

\section{Normalized Difference Vegetation Index}\label{sec:ndvi}

Es gibt eine starke Korrelation zwischen dem physiologischen Status einer Pflanze und deren Chlorophyllgehalt. Faktoren wie Krankheit, Dürre oder Umweltverschmutzung haben einen negativen Einfluss auf den Chlorophyllspiegel.\cite{ref:hendry} Messungen haben ergeben, dass es eine Verbindung zwischen dem Reflexionsgrad im nahen Infrarotbereich und im Rotbereich und dem Chlorophyllgehalt gibt. Das heißt, dass eine gesunde, adulte Pflanze im nahen Infrarotbereich stärker reflektiert als zum Beispiel eine pathologisch veränderte Pflanze. Jedoch bleibt die Reflexion im roten Lichtspektrum in beiden Fällen vergleichsweise schwach. Andere vegetationsfreie Oberflächen wie Acker, Straßen oder Wasser strahlen auch im nahen Infrarotbereich schwach zurück. Dadurch ergibt sich eine zerstörungsfreie Methode, mit einer Multispektralkamera die Vitalität ("`Grünheit"') einer oder mehrerer Pflanzen zu bestimmen.\cite{ref:anatoly} \\\\
Eine multispektralen Aufnahme kann mithilfe der Formel 
\begin{equation}\label{equation:ndvi}
	NDVI = \frac{Band_{NIR} - Band_{Red}} {Band_{NIR} + Band_{Red}}
\end{equation}
dazu genutzt werden, den \textit{Normalized Difference Vegetation Index} (NDVI) zu berechnen. Wobei $Band_{NIR}$ der nahe Infrarotbereich (Near Infrared) und $Band_{RED}$ der sichtbare rote Bereich des elektromagnetischen Spektrums ist. Der NDVI gibt quantifizierte Werte im Bereich von $-1$ bis $1$ zurück. Dabei deuten Werte, die kleiner als $0$ sind, auf Wasserobflächen hin. $0$ bedeutet keine Vegetation. Bei Werte nahe $0$ handelt es sich um spärliche oder ungesunde Vegetation. Das bedeutet je näher ein Wert an $1$ ist, desto dichter bewachsen und gesünder ist die beobachtete Vegetationsfläche.\cite{ref:nasa} Dass bei einem niedrigen, positiven NDVI nicht unterschieden werden kann, ob eine Fläche kaum bewachsen ist oder ungesunde Vegetation besitzt, kann hier vernachlässigt werden. Das Gebiet, das in dieser Arbeit untersucht wird, ist ein bewachsenes Feld, so kann man geringe Vegetation ausschließen.

\section{Sentinel-2}\label{sec:sentinel2}

Die Sentinel-2-Satelliten sind eine von sechs Satellitenarten (Sentinel-1 bis -6) des Copernicus-Programms\footnote{Das Copernicus-Programm wurde von der Europäischen Union zur Erdbeobachtung ins Leben gerufen. Die gesammelten Daten werden für wissenschaftliche, wirtschaftliche und private Anwendungszwecke zur Verfügung gestellt.\cite{ref:copernicus}}, die zur Erdbeobachtung in einen 786 km hohen sonnensynchronen Orbit gebracht wurden. Die Instrumente der Sentinel-2-Satelliten können Aufnahmen in Bereichen des roten und nahen Infrarot- bis hin zum Kurzwelleninfrarotspektrums. Die Aufnahmen haben Gesamtgröße von $100*100$ km und je nach Band eine von Auflösung von 10m, 20m oder 60m (s. Tabelle \ref{tab:resolutions}).

\begin{table}[ht]
\centering
\begin{tabular}{c|c|c|c}
Bandnummer & Auflösung & Wellenlänge (nm) & Bandbreite (nm) \\
\hline
B1 & 60 & 443,9 & 27\\
B2 & 10 & 496,6 & 98\\
B3 & 10 & 560 & 45\\
B4 & 10 & 664,5 & 38\\
B5 & 20 & 703,9 & 19\\
B6 & 20 & 740,2 & 18\\
B7 & 20 & 782,5 & 28\\
B8 & 10 & 835,1 & 145\\
B8a & 20 & 864,8 & 33\\
B9 & 60 & 945 & 26\\
B10 & 60 & 1373,5 & 75\\
B11 & 20 & 1613,7 & 143\\
B12 & 20 & 2202,4 & 242\\
\end{tabular}
\caption{Räumliche und spektrale Auflösungen von Sentinel-2A\cite{ref:sentinel:radiores}}\label{tab:resolutions}
\end{table}
\noindent
Besonders wichtig sind die Bänder B4 (Rot) und B8 (Nahes Infrarot). Mit diesen Bändern kann der NDVI (s. Kapitel \ref{sec:ndvi}) berechnet werden.\cite{ref:sentinel:ndvi} Die Sentinel-2-Satelliten bieten mit $10*10$ m pro Pixel eine hohe räumliche Auflösung.\footnote{Im Vergleich hat zum Beispiel der Landsat-8-Satellit, dessen Daten ebenfalls frei verfügbar sind, eine relativ geringe Auflösung von $30*30$ m.\cite{ref:landsat}} Diese Eigenschaft ist wichtig, um eine mögliche Infizierung genau eingrenzen zu können.\\\\
Dabei ist es auch wichtig, dass die Satelliten regelmäßige Daten liefern können. Durch die gemeinsame Konstellation übertragen die Plattformen alle fünf Tage Daten über einen spezifischen Punkt auf der Erdoberfläche.\cite{ref:sentinel:resolutions} Damit ist gewährleistet, dass der Feldbesitzer ohne persönliche Inspektion ein bis zweimal in der Woche eine Gesundheitseinschätzung über seine Felder erhält.

\section{Das trainierbare Modell}\label{sec:maskrcnn}
In Kapitel \ref{sec:ndvi} und \ref{sec:sentinel2} wurde erklärt wie Daten über die möglichen Erkrankungen geliefert und verarbeitet werden können. Auf den zugrunde liegenden Bilddaten soll nun ein künstliches neuronales Netzwerk (KNN) trainiert werden. In diesem Kapitel wird darauf eingegangen, welche Anforderungen an das KNN gestellt werden, warum das Titel gebende Netz ausgewählt wurde und wie dieses funktioniert.

\subsection{Anforderungen}\label{sec:sub:requirements}
Das KNN muss in der Lage sein, wahrscheinliche Krankheiten in der zu untersuchenden Agrarfläche möglichst genau eingrenzen und klassifizieren zu können. Das ist besonders wichtig, wenn ein Feld von multiplen Krankheiten betroffen ist.\\\\
Es ist damit zu rechnen, dass Daten unter bewölkten Bedingungen aufgenommen werden. Nach starken Niederschlägen können Acker teils oder gänzlich überflutet sein.\cite{ref:root-rot} Das sorgt selbst unter wolkenfreien Bedingungen für einen niedrigen NDVI, obwohl die Nutzpflanzen gesund sind. Das neuronale Netz muss mit solchen \glqq Ausreißern\grqq{} umgehen können.
\\\\
Daraus ergeben sich folgende Kriterien für das neuronale Netzwerk:

\begin{itemize}
	\item Erkennung auf Pixelebene
	\item Robustheit
	\item Hohe Genauigkeit 
\end{itemize}

\subsection{Grundlagen}\label{sec:sub:basics}

\subsubsection{Vollständig vernetztes neuronales Netz}

\noindent
Künstliche neuronale Netze sind mathematische Modelle, die nach dem Vorbild von biologischen neuronalen Netzen gebildet worden sind. So ist ein KNN ebenfalls eine Verbindung von künstlichen Neuronen. Diese Neuronen sind in Schichten angeordnet und jede die Neuronen einer Schicht sind mit den Neuronen nächsten bzw. letzten Schicht verbunden. Zwischen der ersten und der letzten sog. Ausgangsschicht existieren $n$ versteckte Schichten (engl.: hidden layers). 

\begin{figure}[ht]
  \centering
  \includegraphics[width=0.8\textwidth]{pics/neural-net.PNG}
  \caption[Künstliches neuronales Netz]{Künstliches neuronales Netz\cite{ref:verrelst}}
  \label{fig:ann}
\end{figure}
\noindent
Ein Neuron besitzt mehrere Eingangsverbindungen (Gewichte) und ein Ausgangsneuron. Ob ein Neuron "`feuert"', wird durch eine lineare oder nicht-lineare Aktivierungsfunktion bestimmt. Die Eingangsgewichte sind veränderbare Werte, die je nach Höhe einen starken oder niedrigen Einfluss auf die Aktivierungsfunktion haben.
\begin{equation}\label{equ:neuron}
x^{l+1}_{j}=f(\sum\nolimits_i w^l_{ij}x^l_i + w^l_{bj})
\end{equation}
beschreibt das Neuron $j$ in Schicht $l+1$, wobei
\begin{itemize}
	\item $w^l_{ij}$ die Gewichte sind, die Neuron $i$ in Schicht $l$ mit Neuron $j$ verbinden.
	\item $w^l_{bj}$ der Biasterm des $j$-ten Neurons in Schicht $l$ ist.
	\item $f$ die Aktivierungsfunktion ist.\cite{ref:verrelst}
\end{itemize} 

\subsubsection{Convolutional Neural Networks}\label{sec:sub:sub:cnn}
\begin{figure}[ht]
  \centering
  \includegraphics[width=0.95\textwidth]{pics/cnn.png}
  \caption[CNN]{Architektur eines Convolutional Neural Network\cite{ref:cnn-architecture}}
  \label{fig:cnn-architecture}
\end{figure}

\noindent 
\textit{Convolutional Neural Networks} (CNN, dt.: faltendes neuronales Netzwerk) sind Kategorien von neuronalen Netzen, die besonders in der \textit{Computer Vision} Anwendung finden. In der ersten Schicht werden mehrere Merkmale (engl.: features) durch Filter extrahiert und in separate sog. \textit{Feature Maps} abgelegt, um größere Abstraktionsebenen zu erreichen. Diese Filter sind mathematisch mit Faltungen (engl.: convolutions) zu vergleichen und geben dem Netz den Namen. 
\\\\
Die Dimensionen der Feature Maps werden in einem Poolingschritt\footnote{Es gibt verschiedene Arten von Pooling (Max, Average, Sum, ...). Dabei wird die $m*m$ px große Feature Map in sich angrenzende $n*n$ px große Felder eingeteilt ($n<m$). Im Falle von Max-Pooling wird der höchste Wert aus dem Feld übernommen.} (oder auch \textit{subsampling}) reduziert. Dadurch bleiben nur relevante Informationen erhalten und das CNN wird bis zu einem gewissen Grad robust gegenüber Translationen und Rotationen. In der Regel werden die Faltungen und das das Pooling zwei Mal durchgeführt, wie es in Abb. \ref{fig:cnn-architecture} abgebildet ist.
\\\\
Nach der Merkmalextraktion werden die Feature Maps zur Klassifikation in eine eindimensionale Schichten geglättet. Die folgenden Schichten bis zur Ausgangsschicht sind vollständig vernetzt.


\subsection{Mask R-CNN}\label{sec:sub:mask-rcnn}

Im Rahmen dieser Arbeit wird das \textit{Mask Region-based Convolutional Neural Network} untersucht. Mask R-CNN ist eine von Facebook AI Research (FAIR) entwickelte Erweiterung des \textit{Faster R-CNN} und kann verschiedene Instanzen einer Klasse in einem Bild von einander trennen. Dazu muss zuerst die Begriffe der Instanzsegmentierung definiert werden.
\\\\
\begin{figure}[ht]
  \centering
  \includegraphics[width=0.8\textwidth]{pics/instance-segmentation.png}
  \caption[Instanzsegmentierung]{Unterschied Klassifizierung / semantische Segmentierung / Objekterkennung / Instanzsegmentierung\cite{ref:matterport:maskrcnn}}
  \label{fig:instance-segmentation}
\end{figure}
\noindent
Einfache Klassifizierung (engl.: classification) ordnet Bilder als Ganzes einer Klasse zu. \textit{Semantische Segmentierung} (engl.: semantic segmentation) beschreibt die Klassifizierung auf Pixelebene. Es wird erkannt zu welcher Klasse eine Menge von Pixeln gehören, aber es wird nicht zwischen einzelnen Objekten unterschieden. \textit{Objekterkennung} (engl.: object detection) entdeckt und lokalisiert unterschiedliche Objekte, indem es eine Bounding Box um jedes erkannte Objekt zieht. Jedoch fehlt hier die pixelgenaue Abgrenzung einzelner Objektinstanzen. \text{Instanzsegmentierung} (engl.: instance segmentation) kombiniert \textit{Objekterkennung} und \textit{semantische Segmentierung} und ist so in der Lage zwischen einzelnen Objekten zu unterscheiden und ihnen entsprechende Pixel zuzuordnen (s. Abb. \ref{fig:instance-segmentation}) und ist eine der größten Herausforderungen in der Bildverarbeitung.\cite{ref:maskrcnn}
\\\\
\begin{figure}[ht]
  \centering
  \includegraphics[width=\textwidth]{pics/maskrcnn-archtecture.PNG}
  \caption[Mask R-CNN-Architektur]{Mask R-CNN-Architektur\cite{ref:mask-rcnn-architecture}}
  \label{fig:maskrcnn-architecture}
\end{figure}
\noindent
Mask R-CNN ist wie Faster R-CNN in zwei Segmente eingeteilt. In dem ersten Segment, dem \textit{Region Proposal Network} (oder auch RPN), werden mehrere Rahmen (engl.: Bounding Boxes) innerhalb eines Bildes vorgeschlagen, die interessante Objekte beinhalten könnten. Das RPN erzeugt Rechtecke - sog. Anker (engl.: Anchors) - von unterschiedlichen Größen und Bildverhältnissen, die sich über die Bildregion verteilen und sich überlappen. Für jeden Anker wird eine Ankerklasse und eine Bounding-Box-Verfeinerung ausgegeben. Die Klasse unterscheidet Vordergrund und Hintergrund, wobei eine Bounding-Box mit Vordergrundklassifizierung als potentielle Objekterkennung gewertet wird. Ein Anker ist möglicherweise nicht genau über ein Objekt zentriert. Die Verfeinerung ist eine geschätze Veränderung des Ankers in Position, Höhe und Größe, um besser das Objekt umrahmen zu können. Wenn mehrere Anker sich zu sehr überschneiden, wird der Anker mit der höchsten Wahrscheinlichkeit ein Objekt zu beinhalten übernommen und der restlichen Anker werden verworfen.\footnote{Diese Methode wird \textit{Non-max suppression} genannt.}\cite{ref:matterport:maskrcnn}\cite{ref:faster-r-cnn} Die vorgeschlagene Regionen, die einzeln von CNNs bewertet werden, ist der Kernansatz von R-CNN. Das RPN wurde identisch von Faster R-CNN für Mask R-CNN übernommen.\cite{ref:maskrcnn} 
\\\\
\begin{figure}[ht]
  \centering
  \includegraphics[width=.45\textwidth]{pics/rpn-anchors.png}
  \includegraphics[width=.45\textwidth]{pics/rpn-refinement.png}
  \caption[RPN-Anker]{Links: Vereinfachte Darstellung von Ankern über ein Bild\cite{ref:matterport:maskrcnn} / Rechts: Drei Anker (gepunktet), die das das gleiche Objekt umschließen und die Verfeinerung (durchgezogen), die auf diese angewendet wird, um das Objekt genauer einzugrenzen\cite{ref:matterport:maskrcnn}}
  \label{fig:rpn}
\end{figure}
\noindent
Im zweiten Segment werden aus den Regionen \textit{Bounding Boxes} (dt.: Rahmen) und Masken generiert und klassifiziert. Die Rahmen haben verschiedene Größen und können Probleme bei der Klassifizierung verursachen. Daher werden die Rahmen auf eine kleine Feature Map gleicher Größe (z.B. $7*7$ px) reduziert. Die Authoren von \cite{ref:maskrcnn} schlagen eine Methode namens \textit{RoI-Align} vor, bei der Proben aus der Feature Map entnommen werden und eine bilineare Interpolation angewendet wird. In dem bei Faster R-CNN angewandten Verfahren \textit{RoI-Pooling} entstehen durch Quantisierung Informationsverluste und räumliche Abweichungen zwischen Bounding Box und Feature Map, was negative Auswirkungen auf die Maskengenerierung haben kann.\cite{ref:maskrcnn}
\\\\
\begin{figure}[ht]
  \centering
  \includegraphics[width=0.90\textwidth]{pics/fcn-architecture.PNG}
  \caption[FCN-Architektur]{FCN-Architektur\cite{ref:mask-rcnn-architecture}}
  \label{fig:fcn-architecture}
\end{figure}
\noindent
Die oberen vollständig vernetzten Schichten (\textit{FC Layers} in Abb. \ref{fig:maskrcnn-architecture}) klassifizieren die Regionen und die Bounding Boxes berechnet. Dieser Zweig ist für die Objekterkennung wichtig und noch mit Faster R-CNN gemeinsam. 
\\\\
Gleichzeitig werden in einem parallelen Zweig je Bounding Box $k$ $m*n$ große Masken zur semantischen Segmentierung erzeugt, wobei $k$ die Anzahl der Klassen ist.  Anders als in dem ersten Zweig des zweiten Segmentes werden die Masken durch \textit{fully convolutional networks} (FCN, dt.: vollständig faltende Netzwerke) prognostiziert. Diese bestehen nur aus faltenden Schichten, wie sie in Kapitel \ref{sec:sub:sub:cnn} beschrieben sind. Eine Maske ist eine räumliche Kodierung eines Objektes und daher ist es wichtig räumliche Informationen beizubehalten. Diese können durch die Pixel-zu-Pixel-Übereinstimmung extrahiert werden, welche sonst durch vollständig vernetzter Schichten verloren gehen. Diese geben einen Vektor ohne räumliche Dimensionen aus.\cite{ref:maskrcnn}
\\\\
\begin{figure}[ht]
  \centering
  \includegraphics[width=\textwidth, height=5cm]{pics/mrcnn-vs-fcis.PNG}
  \caption[Mask R-CNN vs. FCIS]{Bei FCIS entstehen Artefakte, wenn Objekte sich in einem Bild überlappen.\cite{ref:maskrcnn}}
  \label{fig:maskvsfcis}
\end{figure}
\begin{table}[ht]
  \centering
  \includegraphics[width=\textwidth]{pics/mrcnn-vs-fcis-vs-mnc.PNG}
  \caption[Mask R-CNN im Vergleich]{Instance segmentation \textit{mask} AP auf COCO \textit{test-dev}. MNC und FCIS sind Sieger der COCO 2015 und 2016 Challenge. Mask R-CNN erzielt deutlich bessere Ergebnisse als die komplexere FCIS+++.\cite{ref:maskrcnn}}
  \label{tab:maskvsfcisvsmnc}
\end{table}
\noindent
In \cite{ref:maskrcnn} wird Mask R-CNN mit den \textit{COCO challenge}-Gewinnern\footnote{COCO (Common Objects in Context, dt.: Gewöhnliche Objekte im Kontext) enthält einen Datensatz von über 200000 Bildern in über 80 Kategorien. Der Datensatz ist eine oft genutzte Basis, um Objekterkennungstechniken zu evaluieren und zu bewerten.\cite{ref:coco}} der Jahre 2015 und 2016 verglichen. Der Vergleich zeigt, dass Mask R-CNN in der Challenge bessere Werte erzielt als die Konkurenten (s. Tab. \. Desweiteren fällt \textit{fully convolutional instance segmentation} (FCIS, dt.: vollständig faltende Instanzsegmentierung) auf, wenn es mit überlappenden Objekten konfrontiert wird. Dort erzeugt es Artefakte, welche durch Mask R-CNN nicht entstehen (s. Abb. \ref{fig:maskvsfcis}). Durch diese Gegenüberstellungen wird gezeigt, dass Mask R-CNN alle aufgeführten Anforderungen erzielt. Es erkennt Klasseninstanzen auf Pixelebene und weist eine hohe Robustheit auf. Auch die Genauigkeit hebt sich beim direkten Vergleich ab. Aus diesen Gründen wurde Mask R-CNN im Rahmen diese Arbeit ausgewählt.

\section{Evaluation des Modells}\label{sec:map}

Jetzt wo gezeigt wurde, welches Modell in dieser Arbeit genutzt wird, fehlt eine Möglichkeit ein trainiertes Modell zu bewerten. \textit{Mean average precision} (oder auch mAP) ist eine Metrik, um die Genauigkeit einer Instanzsegmentierung zu messen.\footnote{mAP ist nicht nur auf Instanzsegmentierung limitiert, sondern wird zum Beispiel auch als Metrik in der Objekterkennung genutzt.} Aber bevor die erklärt werden können, muss noch die Begriffe \textit{Precision}, \textit{Recall} und \textit{Intersection over Union} eingegangen werden. 


\subsection{Intersection over Union}

\textit{Intersection over Union} (oder auch IoU, dt: Schnitt über Vereinigung) ist eine wichtige Metrik für die semantische Segmentierung. Sie vergleicht die vorhergesagte Maske mit der Grundwahrheit\footnote{Die Grundwahrheit (engl.: ground truth) ist hier die binäre Maske, die die infizierte Fläche repräsentiert.}, um zu messen wie gut die Vorhersage mit der Grundwahrheit übereineinstimmt.\cite{ref:map}

\begin{equation}\label{equation:recall}
  IoU = \frac{Grundwahrheit\cap Vorhersage}{Grundwahrheit\cup Vorhersage}
\end{equation}
\noindent
Die Schnittmenge beinhaltet alle Pixel, die sich in der Grundwahrheit als auch in der vorhergesagten Maske befinden. Pixel, die sich in der Grundwahrheit und in der Vorhersage befinden, werden von der Vereinigung zusammengefasst. 
\\\\
\begin{figure}[ht]
  \centering
  \includegraphics[width=\textwidth]{pics/iou.PNG}
  \caption[IoU]{Beispiel Intersection over Union, Grundwahrheit in blau, Vorhersage in rot\cite{ref:map}}
  \label{fig:iou}
\end{figure}
\noindent
In der semantischen Segmentierung wird für jede Klasse ein unterschiedlicher IoU-Wert berechnet und dann wird der Mittelwert aus diesen Werten ermittelt, um einen globalen Messwert zu haben. In der Instanzsegmenierung wird für jede einzelne Objektinstanz mittels Instanzgrundwahrheit und Instanzvorhersage ein separater IoU-Wert berechnet. Wenn ein bestimmter Grenzwert überschritten wird, gilt diese Instanz als tatsächlich richtige Erkennung.\cite{ref:jordan}

\subsection{Precision und Recall}

\textit{Precision} (oder auch Falsch-Positiv-Rate) sagt aus mit welcher Wahrscheinlichkeit eine Vorhersage korrekt ist. Diese Metrik wird durch die Formel 
\begin{equation}\label{equation:precision}
  Precison = \frac{RP}{RP + FP}
\end{equation}
berechnet, wobei $RP$ (Richtig-Positiv) die Anzahl der richtigen Erkennungen und $FP$ (Falsch-Positiv) die Anzahl der falschen Erkennungen sei.\cite{ref:map} \textit{Precision} ist also der Anteil von tätsächlich richtigen Erkennungen in Relation zu allen Erkennungen. In Bezug auf Instanzsegmentierung wird die Frage beantwortet, wie viele der erkannten Objekte in einem Bild tatsächlich eine passende Grundwahrheitüberschneidung und eine IoU-Grenzwertüberschreitung haben.\cite{ref:jordan}
\\\\
\textit{Recall} (oder auch Falsch-Negativ-Rate) misst die Wahrscheinlichkeit, dass alle tatsächlich wahren Detektionen korrekt erkannt wurden. Diese Metrik wird durch die Formel 
\begin{equation}\label{equation:recall}
  Precison = \frac{RP}{RP + FN}
\end{equation}
berechnet, wobei $RP$ (Richtig-Positiv) die Anzahl der richtigen Erkennungen und $FN$ (Falsch-Negativ) die Anzahl der Objekte, die fälschlicherweise nicht erkannt wurden, sei. \textit{Recall} ist also der Anteil von tätsächlich richtigen Erkennungen in Relation zu allen Objekten im Datensatz.\cite{ref:map} In Bezug auf Instanzsegmentierung wird die Frage beantwortet, wie viele der Objekte mit Grundwahrheit in einem Bild als tatsächlich richtig erkannt werden und eine IoU-Grenzwertüberschreitung haben.\cite{ref:jordan}

\subsection{Average Precision}

\begin{figure}[ht]
  \centering
  \includegraphics[height=5cm]{pics/kites.jpg}
  \hspace{.5cm}
  \includegraphics[height=5cm]{pics/precision-recall-kurve.png}
  \caption[Precision-Recall-Kurve]{Links: Beispielbild mit multiplen Detektionen und Klassen\cite{ref:arlen}\\Rechts: Beispiel Precision-Recall-Kurve für die Klasse ``Person''\cite{ref:huang}}
  \label{fig:precision-recall}
\end{figure}
\noindent
Ein \textit{Precision}- und \textit{Recall}-Wert bezieht sich jeweils auf eine detektierte Objektinstanz einer Klasse. Bei mehreren detektierten Objekte einer Klasse in einem Bild können diese in einer Precision-Recall-Kurve visualisiert werden (s. Abb. \ref{fig:precision-recall}). \textit{Average Precision} (oder auch AP) fasst die Form der Kurve zu einem Wert zusammen, indem es den Durchschnitt der \textit{Precision}-Werte an elf \textit{Recall}-Werten $[0, 0.1, \dots, 1]$ berechnet: 

\begin{equation}\label{equation:ap}
  AP = \frac{1}{11} \sum_{r \in \{0, 0.1, \dots, 1\}}  p_{interp}(r)
\end{equation}
\noindent
Ein \textit{Precision}-Wert $p$ an der \textit{Recall}-Stelle $r$ wird interpoliert, indem der Maximumwert übernommen an der \textit{Recall}-Stelle $\tilde{r}\ge r$ wird:

\begin{equation}\label{equation:pinterp}
  p_{interp}(r) = \max_{\tilde{r}:\tilde{r}\ge r} p (\tilde{r})
\end{equation}
\noindent
wobei $p(\tilde{r})$ der \textit{Precision}-Wert $p$ an der \textit{Recall}-Stelle $\tilde{r}$ sei. Die Interpolation reduziert den Einfluss kleiner, lokaler Unebenheiten in der Kurve.\cite{ref:huang}

\subsection{Mean Average Precision}

\textit{Mean Average Precision} ist der Durschschnitt aller \textit{Average Precision}-Werte jeder Klasse in jedem Element eines (Sub-)Datensatzes.\footnote{\textit{mAP} wird oft nur \textit{AP} genannt.} \textit{mAP} wird zum Beispiel auch in der COCO- oder PASCAL-VOC-Challenge benutzt, um die Resultate der Challenge-Teilnehmer zu bewerten (s. Tabelle \ref{tab:maskvsfcisvsmnc}). Aber hier kann es zu Unterschieden kommen, wie der \textit{mAP} berechnet wird. So ist es bei der COCO-Challenge der durchschnittliche \textit{mAP} über verschiedene \textit{IoU}-Grenzwerte. Hier wird jeweils ein \text{mAP} an zehn verschiedenen \textit{IoU}-Werten $[0.5, 0.55, \dots, 0.95]$ berechnet und aus den Ergebnissen wird der Durchschnitt ermittelt.\cite{ref:coco:eval} In dieser Arbeit wird stets $IoU=0.5$ als Grenzwert benutzt, um die Auswertung einfach zu halten.
\chapter{Overfitting}\label{chap:overfitting}

\section{Begriffserklärung}\label{sec:what-is-overfitting}

Genaue Daten über Krankheitsbefälle im Agrarsektor sind rar, da diese in der Regel nicht öffentlich zugänglich sind.\footnote{Datenschutz kann ein Grund dafür sein.}  Daher musste mit \textit{Overfitting} gerechnet werden. Das künstliche neurale Netzwerk soll daraufhin trainiert werden, dass es möglichst alle Befälle, die untersucht werden, erkennt. Dafür wird es im ersten Schritt mit einem Trainingsdatensatz trainiert. Im folgenden Schritt mit einem kleineren Validierungsdatensatz überprüft, wie gut das Netz trainiert wird. Overfitting tritt auf, wenn das Netz auf die Daten aus dem Trainingsdatensatz mit sehr hoher Erfolgsquote erkennt, jedoch vergleichsweise schlechte Ergebnisse bei der Validierung bzw. bei unbekannten Daten erzielt. 
\\\\
\begin{figure}[ht]
  \centering
  \includegraphics[height=2.5cm]{pics/mask.png}
  \includegraphics[height=2.5cm]{pics/pred.png}
  \includegraphics[height=2.5cm]{pics/bad-pred.png}
  \caption[Beispiel Overfitting]{V.l.n.r. Bild von infizierter Agrarfläche aus Trainingsdatensatz / Binärmaske der inifizierten Region, wird gemeinsam mit dem linken Bild zum Training in das KNN gespeist / Selbiges Bild, Ergebnis nach Trainingsdurchlauf, prognostizierte Ergebnisfläche in rot / Bild der selben Fläche, was nicht aus dem Trainingsdatensatz stammt, prognostizierte Ergebnisfläche in rot}
  \label{fig:example-overfitting}
\end{figure}
\noindent
\todo{Erwähnen, dass Mask RCNN Bild und Masken für Training benötigt.}
In Abb. \ref{fig:example-overfitting} ist ein Beispiel wie Overfitting sich auswirken kann. Die linken zwei Bilder sind ein exemplarischer Auszug aus dem Trainingsdatensatz. Einmal eine visuelle Repräsentation der NDVI-Werte der infizierten Agrarfläche und die Binärmaske, welche die infizierte Fläche markiert. Das selbe Bild wurde nach einem erfolgreichen Trainingsdurchlauf der Mask R-CNN-Implementierung übergeben und es hat den erkrankten Bereich nahezu perfekt erkannt. Das vierte Bild zeigt zentriert das selbe Feld. Jedoch ist der Ausschnitt größer, rotiert und die Aufnahme stammt von einem anderen Datum.\todo{Genaues Datum nötig?} Der Prognose zur Folge ist die Infizierung auf die benachbarten Felder übergesprungen, was nicht der Wahrheit entspricht. Overfitting ist ein bekanntes Problem im Bereich des maschinellen Lernens und es existieren multiple Methoden, um dem entgegenzuwirken.
\chapter{Konzept und Implementierung}\label{chap:concept}

\section{Konzept}\label{sec:concept}

\begin{figure}[ht]
  \centering
  \includegraphics[width=.2\textwidth]{pics/overview.PNG}
  \caption{Gesamtablauf der Anwendung}
  \label{fig:overview}
\end{figure}
\noindent
Das Programmablauf wird in einzelne Schritte unterteilt, auf die in den nächsten Kapiteln näher eingegangen werden.
\\\\
Zuerst müssen die Daten für das Training bzw. für die Erkennung manuell annotiert werden. Diese Metadaten werden dann genutzt, um automatisch Sentinelprodukte\footnote{Aufnahmenpakete der Sentinel-Plattformen werden als Produkte bezeichnet.} mittels einer API, die von der Copernicus zur Verfügung gestellt wird, herunterzuladen. Aus den Produkten werden die relevanten Bänder extrahiert und unter anderem die jeweiligen NDVI-Werte berechnet. Nachdem die Produkte für das Training vorbereitet wurden, werden die Daten in ein Trainings- und in ein Validierungsdatensatz aufgeteilt. Der folgende Trainingsprozess basiert auf diesen Datensätzen. Sobald das Training abgeschlossen ist, kann die Performanz des Modells getestet werden. 

\section{Annotation}\label{sec:annotation}

Zu Beginn werden die Regionen, die entweder für das Training benutzt oder überprüft werden, manuell erfasst. Vorrausgesetzte Informationen sind
\begin{itemize}
	\item Geografische Koordinaten,
	\item Zeitraum des Befalls und
	\item Bezeichnung der Infektion.
\end{itemize}
Als Format dieser Informationen dient \textit{GeoJSON}\footnote{GeoJSON ist eine Erweiterung des JSON-Format und beschreibt geografische Daten und Geometrien. GeoJSON wird durch den RFC7946-Standard definiert.}. GeoJSON enthält nicht nur geografische Daten, sondern ist auch um benutzerdefinierte Eigenschaften (\texttt{properties}) erweiterbar. Die Annotationen sind also GeoJSON-Features, die ein geografisches Polygon mit Metadaten enthalten.

\begin{lstlisting}[language=json,caption={Annotation},captionpos=b]
{
  "type": "Feature",
  "properties": {
    "disease": 1,
    "from": "2018-07-12T13:00:00Z-7DAYS",
    "to": "2018-07-12T13:00:00Z+7DAYS"
  },
  "geometry": {
    "type": "Polygon",
    "coordinates": [[[11.171988617177981,44.574291380353003],
       [11.1726616444942,44.574017992242283],
       [11.17338129910439,44.575068359984279],
       [11.17273129334275,44.575299863118993],
       [11.171988617177981,44.574291380353003]]]
  }
}
\end{lstlisting}
\noindent
\texttt{properties.disease} enthält die eindeutige, nummerische Repräsentation der Klasse bzw. Krankheit, die in dieser Region enthalten ist. Die Zuordnung der nummerischen Werte und des textuellen Bezeichners werden in einer separaten JSON als Schlüssel-Wert-Paare konfiguiert, wobei der Schlüssel nummerisch und der Wert textuell ist. Hier ist, darauf zu achten, dass der Schlüssel $\ge1$ ist, da $0$ der implizite Schlüssel der Mask R-CNN-Implementierung für den Hintergrund ist. Diese Eigenschaft ist nur für das Training von Relevanz.
\\\\
\texttt{properties.from} und \texttt{properties.to} sind jeweils Start- und Endzeitpunkt, in dem nach verfügbaren Sentinelprodukten gesucht werden soll. Das Format der jeweiligen Eigenschaften kann eine der folgenden Formen haben\footnote{Die Formate basieren auf der \texttt{sentinelsat}-Version 0.12.2.}:
\begin{itemize}
	\item \texttt{yyyyMMdd}
	\item \texttt{yyyy-MM-ddThh:mm:ss.SSSZ} (ISO-8601)
	\item \texttt{yyyy-MM-ddThh:mm:ssZ}
	\item \texttt{NOW}
	\item \texttt{NOW-<n>DAY(S)} (oder \texttt{HOUR(S)}, \texttt{MONTH(S)}, usw.)
	\item \texttt{NOW+<n>DAY(S)}
	\item \texttt{yyyy-MM-ddThh:mm:ssZ-<n>DAY(S)}
	\item \texttt{NOW/DAY} (oder \texttt{HOUR}, \texttt{MONTH} usw.) - Der Wert wird entsprechend (z.B. auf den Tag) gerundet.
\end{itemize}
\noindent
Es ist angebracht einen Zeitraum von mehreren Tagen bzw. Wochen zu wählen, da die Sentinel-2-Satelliten keine täglichen Daten liefern und weil eine Infektion typischerweise über einen längeren Zeitraum vorherrscht. Die Zeitspanne ist von der Krankheit abhängig. Hier wurden eine Woche vor und nach dem Aufnahmezeitpunkt genutzt, um nach Produkten zu suchen.

\section{Suche nach Sentinelprodukten}

\begin{figure}[H]
  \centering
  %\includegraphics[height=0.6\textheight, width=.6\textwidth]{pics/get-products.png}
  \includegraphics[height=0.75\textheight]{pics/get-products.png}
  \caption{Ablaufdiagramm Sentineldatenaufbereitung}
  \label{fig:get-products}
\end{figure}

Der \textit{Copernicus Open Access Hub}\footnote{\url{https://scihub.copernicus.eu/}} ermöglicht freien und offenen Zugriff auf Sentinel-Produkte. Die Daten sind sowohl über eine grafische Oberfläche als auch über eine REST-API verfübar. Vorrausgesetzung für beide Optionen ist ein Account, der über die grafische Oberfläche erstellt werden kann.
\\\\
Die Nutzung der Schnittstelle erfolgt über die Python-Bibliothek \texttt{sentinelsat}\footnote{https://sentinelsat.readthedocs.io/en/stable}. Bei einer Anfrage müssen die GeoJSON-Dateien in WKT\footnote{WKT (Well-known text) ist eine Markup-Sprache zur Repräsentation von geometrischen Objekten auf Karten und räumlichen Referenzsystemen.} umgewandelt werden, was von der Bibliothek übernommen werden kann. Die WKT-Geometrie wird als \textit{footprint} (dt.: Fußabdruck) bezeichnet. Solang es nicht anders angegeben wird, gibt die Corpernicus-API Produkte zurück, die die RoI schneiden. Desweiteren wird der Plattformname statisch als `Sentinel-2' definiert, damit keine Produkte von den anderen Sentinelplattformen zurückgegeben werden. Der Suchzeitraum wird aus der jeweiligen GeoJSON-Datei übernommen. Sollten für die Suchanfragen keine Produkte existieren, wird die Anwendung beendet, da es keine Basis gibt, auf der das Netzwerk trainiert werden kann. Eventuell muss bei so einem Fall der Zeitraum erweitert und Prozess wiederholt werden. Bei vorhandenen Produkten lädt das Skript diese herunter. Die Produkte werden in einem komprimierten Format geliefert und enthalten neben zusätzlichen Informationen, Banddaten in separaten Dateien im JPEG2000-Format\footnote{JPEG 2000 genau wie GeoTIFF ist ein Bildformat in dem auch Metadaten abgelegt werden können. So sind Pixel geografischen Koordinaten zuordbar.}.
\\\\
Für jedes Produkt, das zur aktuellen Annotation gehört, wird ein neuer Eintrag zu einer \textit{FeatureCollection} hinzugefügt. Für die spätere Entwicklung sind die Annotationen so leichter zu finden und bearbeitbar. Außerdem bleiben dadurch die Originaldaten unberührt. Dieser Schritt wird für jede vorhandene Annotationsdatei wiederholt. Anschließend werden die Bilddateien für B4 und B8 aus dem Produkt extrahiert.



\section{Aufbereitung der Sentineldaten}

\begin{figure}[H]
  \centering
  \includegraphics[height=0.75\textheight]{pics/create-ndvi.png}
  \caption{Ablaufdiagramm der Aufbereitung}
  \label{fig:create-ndvi}
\end{figure}

Originale Sentinel-2-Aufnahmen haben eine Größe von $10980*10980$ px und müssen deshalb deutlich verkleinert werden, um die Speicherbelastung zu minimieren. Vor allem da nur kleine Ausschnitte von wenigen Pixeln benötigt werden. 
\\\\
Zu Beginn werden alle Elemente aus der vorher erstellten \textit{FeatureCollection} geladen und nacheinander bearbeitet. Jedes Produkt hat ein eigenes Koordinatenreferenzsystem (engl.: Coordinate Reference System, CRS) und unter Umständen unterscheiden sich die Systeme des Produktes und des Polygons\footnote{Nach RFC 7946 ist das geografische Referenzsystem \textit{World Geodetic
 System 1984} (WGS 84) das Standardsystem.\cite{ref:rfc7946}}. Diese müssen gleich sein, um miteinander agieren zu können. Das Produkt-CRS kann aus den extrahierten Bändern gelesen und dann dazu genutzt werden, um die Annotationskoordinaten in eben dieses zu projizieren.
 \\\\
 Nachdem abgesichert wurde, dass die RoI in dem Sentinelprodukt enthalten ist, wird aus den Bändern die RoI samt einem vorher definierten Rand ausgeschnitten. Der Rand ist später bei der Data Augmentation hilfreich. Gleichzeitig erstellt die Anwendung eine gleich große binäre Maske, wobei die Elemente der Maske mit den Pixeln der RoI korrespondieren. Die Elemente, die die RoI repräsentieren, enthalten einen wahren Wert.
 \\\\
\begin{figure}[ht]
  \centering
  \includegraphics[height=3cm]{pics/b4.PNG}
  \includegraphics[height=3cm]{pics/b8.PNG}
  \includegraphics[height=3cm]{pics/ndvi.PNG}
  \caption[B4 - B8 - NDVI]{V.l.n.r. B4 (RED), B8 (NIR), NDVI}
  \label{fig:ndvi}
\end{figure}
\noindent
Nun wird der NDVI aus den B4- und B8-Ausschnitten, wie in Kapitel \ref{sec:ndvi} gezeigt, berechnet (s. Abb. \ref{fig:ndvi}). Das Ergebnis wird nun mittels Data Augmentation vervielfältigt. Dazu wird es vier mal um 90\degree gedreht. Danach wird jedes rotierte Bild horizontal und vertikal gespiegelt. Darauf werden neun mal aus den rotierten und gespiegelten Daten zufällig Bilder in der Größe von $16*16$ px ausgeschnitten. Durch diese Operationen vergrößert sich die Grundgesamtheit um den Faktor 108. Jede Operation wird ebenfalls auf die entsprechende Maske angewandt. Dabei wird darauf geachtet, dass die Maske nicht vollständig aus falschen Werten besteht und dass die manipulierten Dateien ebenfalls der \textit{FeatureCollection} hinzugefügt werden.

\section{Trainings- und Validierungsdatensatz}\label{sec:dataset}

\begin{figure}[ht]
  \centering
  \includegraphics[height=0.5\textheight,width=0.2\textwidth]{pics/create-sets.png}
  \caption{Ablaufdiagramm der Datensatzaufteilung}
  \label{fig:create-datasets}
\end{figure}
\noindent
Die berechneten NDVI-Bilddateien und deren Masken bilden die Gesamtheit des Datensatzes. Um das Modell trainieren und testen zu können, muss diese Gesamtheit aufgeteilt werden. 
\begin{itemize}
	\item Trainingsdatensatz \\
		Hiermit werden die Gewichte des neuronalen Netzwerkes trainiert.
	\item Validierungsdatensatz \\ 
		Dieser Datensatz wird genutzt, um das trainierende Modell wiederholt zu evaluieren. Die Parameter in dem Modell werden auf Basis dieser Evaluation verändert, um das Idealergebnis zu approximieren.
	\item Testdatensatz \\
		Während der Validierungsdatensatz während des Trainings genutzt wird, ist der Testdatensatz zu Evaluation des fertig trainierten Modells geeignet. Das KNN sieht also die Elemente nicht während des Trainingprozesses und repräsentiert "`fremde"' Daten. Damit ist der Testdatensatz ein besserer Indikator der Leistung des Modells als der Validierungsdatensatz.
\end{itemize}

\begin{figure}[ht]
  \centering
  \includegraphics[width=.9\textwidth]{pics/data-split.png}
  \caption[Datensatzverteilung]{Darstellung der relativen Datensatzverteilung\cite{ref:dataset}}
  \label{fig:split}
\end{figure}
\noindent
In welchem Verhältnis die Datensätze aufgeteilt werden, hat einen Einfluss auf das Training und hängt von der eigentlichen Größe des gesamten Datensatz und von dem trainierten Modell ab. Komplexere Modelle benötigen einen größeren Validierungsdatensatz, während ein Modell mit weniger Parametern mit einem kleineren auskommt.\cite{ref:dataset} Typischerweise besteht jedoch der Trainingsdatensatz aus dem größten Teil. 
\\\\
Hier wird zunächst die Gesamtmenge zwischen einem unbdestimmten Teildatensatz ($80\%$ der Gesamtheit) und dem Testdatensatz ($20\%$ der Gesamtheit) aufgeteilt. Danach teilt sich der Teildatensatz in den Trainingsdatensatz ($80\%$ des Teiles) und in den Validierungsdatensatz ($20\%$ des Teiles). Die hier angewandte prozentualen Werte sind häufig angewandte initiale Verhältnisse und können bei Bedarf angepasst werden. Die geteilten Mengen werden unterschiedlichen Orten abgelegt und für jede Sammlung wird jeweils eine neue \textit{FeatureCollection} erzeugt, die die Annotationen der entsprechenden Elemente enthalten. 


\section{Training/Detektion}\label{sec:training}

Mit den vorbereiteten und aufgeteilten Daten kann nun das Netzwerk angelernt werden. In dieser Arbeit wurde eine Implementierung\footnote{\url{https://github.com/matterport/Mask_RCNN}} des kalifornischen Unternehmens Matterport, Inc. erweitert und genutzt. Matterport stellte den Quellcode 2017 unter MIT-Lizenz der Öffentlichkeit zur freien Verfügung. Das neuronale Netz wurde mittels Python 3, Tensorflow und Keras implementiert. Zum Training eigener Daten müssen die zwei Python-Basisklassen \texttt{Dataset} und \texttt{Config} erweitert werden. 

\subsection{Dataset}\label{sub:sec:dataset}

Die Klasse \texttt{Dataset} bietet einen Weg, eigene Datensätze in das Modell zu laden, da die Daten in unterschiedlichen Formaten vorkommen können. Für die Erweiterung erbt die neue Klasse \texttt{CropDiseaseDataset} von \texttt{Dataset} und überschreibt die Methoden \texttt{Dataset.load\_mask}, \texttt{Dataset.load\_image} und \texttt{Dataset.image\_reference}. Zusätzlich wurde die Methode \texttt{CropDiseaseDataset.load\_crop\_disease} implementiert. Eine Instanz von \texttt{Dataset} bzw. \texttt{CropDiseaseDataset} bildet nicht die gesamte Datenmenge ab, sondern jeweils einen spezifizierten Subdatensatz, die in Kapitel \ref{sec:dataset} definiert wurden. So muss jeder Teil separat instanziiert werden. 
\\\\
\texttt{CropDiseaseDataset.load\_crop\_disease} fügt dem Objekt die Klassen- bzw. Bildinformationen wie eindeutige Bezeichnung und Dateipfad durch die internen Methoden \texttt{Dataset.add\_class} bzw. \texttt{Dataset.add\_image} hinzu. 
\\\\
\texttt{CropDiseaseDataset.load\_image} lädt die eigentliche Bilddatei in den Arbeitsspeicher. Die NDVI-Werte wurden als Grauskalenbilder gespeichert und müssen in ein RGB-Format konvertiert werden. Es ist möglich durch Anpassungen des Mask RCNN-Codes, auch Dateien mit mehr oder weniger als drei Farbkanälen zu nutzen. Jedoch verursachte das Fehler, die zum Zeitpunkt der Verschriftlichung nicht gelöst wurde. Da es sich dabei um keinen kritischen Fehler handelt, wurde dieser durch die Konvertierung umgangen. Da die NDVI-Werte zwischen $-1$ und $1$ liegen und Farbwerte aus ganzzahligen Werten bestehen, werden die NDVI-Werte mit $255$ multipliziert, bevor sie schließlich hinzugefügt werden.
\\\\
\texttt{CropDiseaseDataset.load\_mask} liest die binären Masken aus den Dateien und ordnet sie einer Klasse und einem Bild zu. 
\\\\
Die Methode \texttt{CropDiseaseDataset.image\_reference} gibt den vollständigen Dateipfad einer Bilddatei zurück.

\subsection{Config}

Die Klasse \texttt{Config} ist eine Sammlung von Parametern, die zur Konfiguration des Modells genutzt werden. Diese Parameter können entsprechend angepasst werden und haben jeweils Einfluss auf das Training bzw. auf die Erkennung. Es wird hier nur auf die Parameter eingegangen, die explizit im Rahmen der Entwicklung verändert wurden, um das Ergebnis des Trainings zu verbessern. Die Parameterbezeichnung ist der Name des Parameters, so wie er in \texttt{Config} deklariert wurde. Die Erklärung beschreibt den Parameter näher. Die angegebenen Werte sind alle möglichen Werte, die in der Entwicklung genutzt wurden. Welche Werte bei welchem Trainingsdurchlauf genutzt wurden, wird im Kapitel \ref{sec:experiments} beschrieben.
\\\\
\noindent
\textbf{Parameterbezeichnung:} BACKBONE\\
\textbf{Erklärung:} Die Backbone-Architektur mit der das Modell gebildet wird.\\
\textbf{Werte:} resnet50\\

\noindent
\textbf{Parameterbezeichnung:} DETECTION\_MIN\_CONFIDENCE\\
\textbf{Erklärung:} Minimalste Wahrscheinlichkeit einer Detektion, um sich als vorhergesagte Instanz zu qualifizieren. Detektionen mit einer Wahrscheinlichkeit niedriger als dieser Wert werden ignoriert.\\
\textbf{Werte:} 0\\

\noindent
\textbf{Parameterbezeichnung:} GPU\_COUNT\\
\textbf{Erklärung:} Anzahl der Grafikkarten, auf denen das Modell trainiert wird. Wenn die CPU die Berechnungen übernimmt, muss der Parameter den Wert $1$ annehmen.\\
\textbf{Werte:} 1\\

\noindent
\textbf{Parameterbezeichnung:} IMAGE\_MAX\_DIM\\
\textbf{Erklärung:} Wichtig für \texttt{IMAGE\_RESIZE\_MODE}. Bildhöhe und -breite werden auf diesen Wert (in px) vergrößert.\\
\textbf{Werte:} 128\\

\noindent
\textbf{Parameterbezeichnung:} IMAGE\_RESIZE\_MODE\\
\textbf{Erklärung:} Methode mit der ein Bild vergrößert bzw. verkleinert wird. Die gewählte Option vergrößert die Datensätze auf \texttt{IMAGE\_MAX\_DIM}$*$\texttt{IMAGE\_MAX\_DIM} und füllt das Bild mit 0-Werten, sollte das Ausgangsbild kein Quadrat sein.\\
\textbf{Werte:} square\\

\noindent
\textbf{Parameterbezeichnung:} IMAGES\_PER\_GPU\\
\textbf{Erklärung:} Bilder die pro Schritt gleichzeitig für das Training in den Speicher geladen werden. Dieser Wert ist - je nachdem man auf der CPU oder auf der GPU trainiert - abhängig von der Größe der Arbeitsspeichers oder Grafikspeichers. Für die Detektion ist nur ein Bild pro Schritt erlaubt.\\
\textbf{Werte:} $1$, $4$\\

\noindent
\textbf{Parameterbezeichnung:} LEARNING\_RATE\\
\textbf{Erklärung:} Die Lernrate gibt an, wie stark die Gewichte des Netzwerkes korrigiert werden. Wenn dieser Wert zu klein ist, kann die Konvergenz zum Ideal sehr lange dauern. Ein zu hoher Wert kann das Ideal überschießen oder sogar dafür sorgen, dass die Lernkurve divergiert. He et al. nutzen in ihrer Ausarbeitung einen Wert von $0.02$.\cite{ref:maskrcnn} Dieser sorgt jedoch laut den Entwicklern von Matterport zu einem explosionartigen Anstieg der Gewichte, weswegen sie sich für $0.001$ entschieden haben.\cite{ref:matterport} Hier werden beide Werte untersucht.\\
\textbf{Werte:} $0.001$, $0.2$\\

\noindent
\textbf{Parameterbezeichnung:} NUM\_CLASSES\\
\textbf{Erklärung:} Die Anzahl der Klassen, die trainiert werden. Hier ist darauf zu achten, dass der Hintergrund als eigene Klasse gesehen wird, auch wenn diese nicht explizit definiert wird. Daher muss die zusätzliche Klasse mit in diesen Wert einberechnet werden.\\
\textbf{Werte:} 2\\

\noindent
\textbf{Parameterbezeichnung:} RPN\_ANCHOR\_SCALES\\
\textbf{Erklärung:} Quadratgröße der RPN-Anker. Die Werte sollten kleiner sein als \texttt{IMAGE\_MAX\_DIM}. Anker, die größer sind, machen keinen Sinn, da Objektinstanzen sich innerhalb der Bildgrenzen befinden.\\
\textbf{Werte:} (8, 16, 32, 64, 128), (4, 8, 16, 32, 64)\\

\noindent
\textbf{Parameterbezeichnung:} STEPS\_PER\_EPOCH\\
\textbf{Erklärung:} Anzahl der Trainingsschritte bis eine Epoche abgeschlossen ist. Der Wert ist abhängig von der Größe des Trainingdatensatzes und wie viele Bilder pro Schritt prozessiert werden.\\
\textbf{Werte:} $\frac{SIZE_{Train}}{IMAGES\_PER\_GPU}$\\

\noindent
\textbf{Parameterbezeichnung:} USE\_MINI\_MASK\\
\textbf{Erklärung:} Wenn dieser bool'sche Wert wahr ist, werden Instanzmasken zu einer spezifizierten Größe verkleinert. Hier wird das jedoch deaktiviert, da die Eingangsbilder klein genug sind und diese Operation nicht nötig ist.\\
\textbf{Werte:} \texttt{False}\\

\noindent
\textbf{Parameterbezeichnung:} WEIGHT\_DECAY\\
\textbf{Erklärung:} Einflussgröße der L2 Regulization.\\
\textbf{Werte:} $0.01$, $0.005$, $0.001$\\

\noindent
Der Konfigurationsparameter \texttt{BATCH\_SIZE} wird automatisch aus $GPU\_COUNT * IMAGES\_PER\_GPU$ berechnet. Während \texttt{IMAGES\_PER\_GPU} angibt, wie viele Bilder pro Rechnereinheit (GPU oder CPU) in das neuronale Netz geladen werden, definiert \texttt{BATCH\_SIZE} wie viele Bilder insgesamt pro Trainingsschritt geladen werden.
\\\\
\texttt{CropDiseaseConfig} erbt von \texttt{Config} und die Werte werden vor jedem erneuten Trainingsdurchlauf verändert und der Modell-Instanz übergeben.  

\subsection{Ablauf}

Die Benutzer können bei Skriptaufruf als Kommandozeilenparameter spezifieren, ob das neuronale Netz trainiert oder ob ein Bild überprüft werden soll. Für beide Fälle müssen sie spezifizieren, auf welcher Basis das neuronale Netz gebildet werden soll. Hier gibt es drei Optionen für diese Anwendung:

\begin{itemize}
	\item vortrainierte COCO-Gewichte
	\item vortrainierte ImageNet-Gewichte
	\item Fortsetzen eines alten Trainingsdurchlaufs
\end{itemize}
\noindent
Wählen die Benutzer eine der ersten beiden Möglichkeiten, werden die vortrainierten Gewichte heruntergeladen. Bei der dritten Option muss eine vorher gespeicherte Datei geladen werden, die die Gewichte des gewünschten Modells enthält. Aus den gewählten Gewichten und dem \texttt{CropDiseaseConfig}-Objekt wird nun eine Mask R-CNN-Instanz gebaut. Dieses Instanz kann eine von zwei Modi 
\begin{itemize}
	\item training
	\item inference
\end{itemize}
annehmen. Wobei \textit{training} die Gewichte des Modells für das Training veränderbar macht und \textit{inference} für die Detektion die Gewichte einfriert, da sie während einer Erkennung nicht trainiert werden müssen.

\subsubsection{training}

Wie in Kapitel \ref{sub:sec:dataset} beschrieben, ist für jeden Subdatensatz jeweils ein unterschiedliches \texttt{CropDiseaseDataset}-Objekt nötig. In den separaten Objekten werden nun die Daten, Klassen und Masken für das Training, die Validierung und das Testen geladen und vorbereitet.
\\\\
Das Test-Objekt wird nur indirekt für das Training benutzt. Es hat keinen direkten Einfluss auf den Trainingsverlauf, sondern wird genutzt, um den mAP am Ende jeder Epoche zu berechnen. Hierzu ist eine weitere Mask R-CNN-Instanz im \textit{inference}-Modus notwendig, dessen Gewichte nach jeder Epoche auf die selben Werte des \textit{training}-Modells aktualisiert werden. Die Konfiguration definiert die von \texttt{CropDiseaseConfig} abgeleitete Klasse \texttt{InferenceConfig}. Anders als die Parameter von \texttt{CropDiseaseConfig} werden die Werte nur einmalig festgelegt.
\begin{lstlisting}[language=python,caption={InferenceConfig},captionpos=b]
class InferenceConfig(CropDiseaseConfig):
	# Nur ein Bild pro Detektion erlauben
	IMAGES_PER_GPU = 1
	GPU_COUNT = 1
	# Jede Erkennung wird angezeigt
	DETECTION_MIN_CONFIDENCE = 0.0
\end{lstlisting}
Auf Grund des \textit{inference}-Modells und des Testdatensatzes kann nun am Ende einer Epoche der mAP berechnet werden, mit dessen Hilfe am Ende des Trainings das neuronale Netz evauliert werden kann. Um Vergleichswerte zu haben, wird dieser für Trainings- und Validierungsdatensatz ebenfalls berechnet. 
\\\\
Bevor das Training starten kann, müssen noch die Anzahl der Epochen und die trainierbaren Schichten angegeben werden. Die Anzahl der Epochen hat ein Einfluss darauf, wie oft ein Training wiederholt wird. Wird der Wert zu niedrig gewählt, kann es sein, dass die Gewichte nicht ausreichend genug trainiert sind. So riskiert man \textit{Underfitting}. Ist der Wert jedoch zu hoch, werden die Gewichte zu sehr an den Trainingsdatensatz angepasst und ein \textit{Overfitting} ist die Folge. Mit dem \textit{training}-Modus gibt man zwar an, dass die Gewichte variabel sind, aber durch die hier angegeben Schichten wird genau festgelegt, welche Schichten eingefroren werden und welche nicht. So wird mit \texttt{layers='heads'} lediglich der \textit{network head} trainiert, werden alle Schichten mit \texttt{layers='all'} trainierbar sind. Bei einem kleinen Datensatz wäre es zum Beispiel sinnvoll, nur die Klassifikatoren zu trainieren, die sich im Netzwerkkopf befinden. Zusätzlich können verschiedene Schichten iterativ trainiert werden. Zum Beispiel wird in den ersten 20 Epochen der \textit{head} trainiert. Anschließend werden alle Schichten für 80 Epochen trainiert.

\begin{figure}[H]
  \centering
  \includegraphics[height=.95\textheight,width=.4\textwidth]{pics/train-model.png}
  \caption{Ablaufdiagramm des Trainings}
  \label{fig:training}
\end{figure}

\subsubsection{inference}

\begin{figure}[ht]
  \centering
  \includegraphics[height=0.6\textheight]{pics/detect.png}
  \caption{Ablaufdiagramm der Erkennung}
  \label{fig:training}
\end{figure}
\noindent
Im \textit{inference}-Mdous wird das Mask R-CNN-Objekt durch \texttt{InferenceConfig} konfiguriert. Wenn die angebebenen Gewichte in das Modell geladen wurden, können $n$ Bilder, wobei $n\ge 1$ sei, zur Detektion in das Netzwerk gegeben werden, ohne dass eine spezielle \texttt{Dataset}-Instanz vonnöten ist. Es ist wichtig, dass das die NDVI-Werte vor der Detektion zu RGB-Werten konvertiert werden, wie es in Kapitel \ref{sub:sec:dataset} beschrieben ist.
\\\\
Nach der Detektierung mit möglichen Objekten gibt das Modell eine Liste mit $n$ Elementen zurück. Die Elemente dieser Listen bestehen aus $m$ vorhergesagten \textit{RoIs}, Masken und Klassen der jeweiligen Bilder, wobei $m \ge 0$ sei. Diese sind in separaten Listen abgelegt und sind so angeordnet, dass die jeweiligen Elemente der Listen an Position $i$ zusammengehören, wobei $0\le i < m$ sei. So gehören die $i$-te \textit{RoI}, Klasse und Masken zu der selben vorhergesagten Objektinstanz. Zusätzlich liegt der Instanz ein Wert zwischen $0$ und $1$ bei, die die Wahrscheinlichkeit der Korrektheit der Vorhersage angibt. 

% ----- Abbildungen ----- %
\listoffigures


% ----- Tabellen----- %
\listoftables

\pagestyle{plain} % zurueck setzen von roemische seitenanzahl


%\printbibliography
\bibliographystyle{abbrv}
\bibliography{literature}
	
% ----- Listings ----- %

\include{statement}

\end{document}