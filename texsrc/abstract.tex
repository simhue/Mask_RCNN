\begin{abstract}
\noindent
Pathogene verursachen weltweit enorme Einbußen von Ernteerträgen. Zum Beispiel kann aus einer Anthraknose-Infektion ein Ertragsverlust von bis zu 50 \% resultieren. Methoden zur Bekämpfungen und Prävention wie Pestizide von Erkrankungen belasten nicht nur die Landwirte monetär, sondern auch durch die Verteilung von Chemikalien die Umwelt schädigen. Daraus folgende manuelle Kontrollen sind zeitaufwendig. Die Sentinel-2-Satelliten erzeugen regelmäßige Multispektralaufnahmen, aus denen ein Normalized Difference Vegetation Index (NDVI) berechnet werden kann. Der NDVI ist ein Indikator für den Vitalitätsstatus der Nutzpflanzen. In dieser Arbeit wurde das Mask Region-based Convolutional Neural Network (Mask R-CNN) untersucht, ob es eine Nutzpflanzen-Infektion anhand der NDVI-Werte identifizieren kann. Da nur wenige Daten über Infektionen vorhanden sind, wurden ebenfalls Data Augmentation und L2 Regularization zur Vermeidung von Overfitting evaluiert. Es hat sich gezeigt, dass Data Augmentation eine nützliche Methode gegen Overfitting ist. Darüber hinaus hilft es durch Randomisierungen des Datensatzes zur Generalisierung des Mask R-CNN. Obwohl Mask R-CNN und Data Augmentation potentielle Werkzeuge sind, um Agrarflächen zu analysieren, sind Forschungen mit weiteren Daten von erkrankten Feldern notwendig.
\end{abstract}